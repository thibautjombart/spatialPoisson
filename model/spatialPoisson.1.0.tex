\documentclass[a4paper,11pt]{article}
\usepackage[utf8]{inputenc}
\usepackage{amsfonts}
%opening
\title{A spatial Poisson transmission model for Ebola (and other diseases)}
\author{Thibaut, Pierre, Anne and the Ebola team}

\begin{document}

\maketitle
% 
\begin{abstract}
The model is a meta-population model using a known (spatial) connectivity matrix between patches and a simple kernel to model dispersal. 
The model is based on incidence data, with only infected individuals being known.

\end{abstract}

\newpage
%%%%%%%%%%%%%%%%%%%%%%%
%%%%%%%%%%%%%%%%%%%%%%%
\section{Notations}
%%%%%%%%%%%%%%%%%%%%%%%
%%%%%%%%%%%%%%%%%%%%%%%

\begin{itemize}
 \item $I_t^i$: incidence in patch $i$ at time $t$ (data)
 \item $I_t$: vector of incidence of all patches $(I_t^1, \ldots, I_t^N)$
 \item $T$: the last date of the data
 \item $D_{ij}$: distance between $i$ and $j$
 \item $w(.)$: the known probability mass distribution of the generation time / serial interval
 \item $N$: number of patches in the model
 \item $n_t^i$: the number of infected individuals in patch $i$ at time $t$
 \item $d_{j\rightarrow i}$: intensity of dispersion from $j$ to $i$
 \item $\delta$: general dispersal parameter
 \item $R$: the effective reproduction number
 \item $t_k$: the date of infection of individual $k$
 \item $f_\mathcal{P}(a,b)$: the Poisson density for $a$ observations and a rate $b$
 \item $k(c,d)$: a spatial kernel for a distance $c$ and a parameter $d$
\end{itemize}





%%%%%%%%%%%%%%%%%%%%%%%
%%%%%%%%%%%%%%%%%%%%%%%
\section{Model}
%%%%%%%%%%%%%%%%%%%%%%%
%%%%%%%%%%%%%%%%%%%%%%%

$I_t^i$ is assumed to follow a Poisson distribution of parameter $\lambda_t^i$.
$\lambda_t^i$ is a sum over of the forces of infection of all patches towards $i$ (including $i \rightarrow i$). 
We note $\beta_t^i$ the force of infection coming from infected individuals in patch $i$ at time $t$, defined as:
\begin{equation}
 \beta_t^i = R \sum_{k=1}^{n_t^i} w(t - t_k)
\end{equation}
The force of infection experienced by patch $i$ at time $t$ is then:
\begin{equation}
\lambda_t^i =  \sum_{j=1}^N d_{j\rightarrow i} \beta_t^i
\end{equation}
with:
\begin{equation}
d_{j\rightarrow i} = k(D_{ij}, \delta)
\end{equation}

The likelihood of $I_t$ is defined as:
\begin{equation}
p(I_t | R, \delta) = \prod_{i=1}^N f_\mathcal{P}(I_t^i, \lambda_t^i)
\end{equation}

By extension, the likelihood for the entire data:
\begin{equation}
p(I_1, \ldots, I_t | R, \delta) = \prod_{t=1}^T \prod_{i=1}^N f_\mathcal{P}(I_t^i, \lambda_t^i)
\end{equation}


\end{document}
